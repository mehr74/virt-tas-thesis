%!TEX root = ../Thesis.tex



%---------------------------------------------------------------------------------------------%
% a paragraph about virtualization, that is widely used
Virtualization has gain a lot of popularity during the last two decades. Enabling fast 
provisioning, increasing the availability time, and reducing the maintenance costs
are a few reasons for the gained popularity of virtualization. Moreover, by multiplexing 
multiple virtual machines (VMs) on a same physical server less enegry gets consumed, 
which makes virtualization enviromental friendly and even more popular thesedays.

%---------------------------------------------------------------------------------------------%
% a paragraph about networking in virtualized environment
Public cloud providers such as Amazon Web Services (AWS), Microsoft Azure, and Google Cloud
Platform (GCP) are expanding the range of workloads which can be spilled over public cloud. 
These  vendors, in addition to providing high number VMs with increased performance, have made 
it possible for public tenants to migrate their home workloads without changing the 
on-premises networking configurations. Furthermore the users of these environments
benefit from supplied address spaces, security groups and ACLs, scalable load balancers, 
bandwidth metering, QoS, and more \cite{firestone2017vfp}.


%---------------------------------------------------------------------------------------------%
% about new stacks for processing tcp
Applications residing on VMs require high throughput and low latecy network access.

%---------------------------------------------------------------------------------------------%
% a graph of numbers which motivates us to provide new stacks for applications


%---------------------------------------------------------------------------------------------%
% about netkernel and why it does not the problem we are solving


%---------------------------------------------------------------------------------------------%
% we asked ourself whether it is possible to provide


%---------------------------------------------------------------------------------------------%
% briefly explain how VirtTAS works


%---------------------------------------------------------------------------------------------%
% what are the main contributions of this thesis


%---------------------------------------------------------------------------------------------%
% structure of next sections


