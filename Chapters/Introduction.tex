%!TEX root = ../Thesis.tex

%---------------------------------------------------------------------------------------------%
% a paragraph about virtualization, that is widely used
Virtualization has gain a lot of popularity during the last two decades. Enabling fast 
provisioning, increasing the availability time, and reducing the maintenance costs
for users are a few reasons of the gained popularity. Moreover, by multiplexing 
multiple virtual machines (VMs) on a same physical server less enegry gets consumed, 
which makes virtualization enviromental friendly and even more popular thesedays.

%---------------------------------------------------------------------------------------------%
% a paragraph about networking in virtualized environment
Public cloud providers such as Amazon Web Services (AWS), Microsoft Azure, and Google Cloud
Platform (GCP) are expanding the range of workloads which can be spilled over virtualized environments. 
These  vendors, in addition to providing high number of VMs with increased performance, have made 
it possible for public tenants to migrate their home workloads without changing the 
on-premises networking configurations. Furthermore, the users of these environments
benefit from supplied address spaces, security groups and ACLs, scalable load balancers, 
bandwidth metering, QoS, and more \cite{firestone2017vfp}.


%---------------------------------------------------------------------------------------------%
% about new stacks for processing tcp
The workloads applied by cloud tenants are bursty, which require high throughput and low latecy 
network access. It was this requirement, which motivated cloud providers to increase their networking 
speeds by more than 40x and more in a span of only a few years \cite{firestone2018azure}. While 
networking speed is growing fast, the CPU improvement has become slower, and has experienced the 
end of Moore's law \cite{esmaeilzadeh2011dark}. Hence efficient packet processing is becoming 
more and more important. 

Due to poor cache allocation, costly context switches, and resources sharing across different 
cores, Linux TCP packet processing is considered as inefficient for modern data center 
networking \cite{kaufmann2019tas, shashidhara2022flextoe}. Different techniques tried 
to address the issue by reducing the overheads and improving the conventional TCP stack.
These techniques include bypassing kernel to enable direct NIC access from user-space
\cite{belay2014ix, jeong2014mtcp, prekas2017zygos}
using receive side scaling (RSS) to carefuly steering and processing packets on multi-cores 
architecture \cite{marty2019snap, kaufmann2019tas}, and offloading packet processing to 
NICs with computation capabilities 
\cite{arashloo2020enabling, lin2020panic, firestone2018azure, shashidhara2022flextoe}. 


%---------------------------------------------------------------------------------------------%
% why modern tcp stacks are not applicable by VMs
Although these techniques have imporved the performance of end-host packet processing, most cloud 
operators hesitate to deploy them on their fleets. Majority of these techniques are proposed only for 
applications running on bare metal servers and they do not consider providing the rich 
virtualization features required in multi-tentant datacenters. 

%---------------------------------------------------------------------------------------------%
% we asked ourself whether it is possible to provide
% TODO extend 
In this thesis we asked the following question: how can we provide virtualization features on 
top of the modern stacks? and we tried to come up with an implementation which shows how we 
do. To do so, we chose TAS as bla bla, and Open vSwitch as bla bla. We integrate the fast path.

%---------------------------------------------------------------------------------------------%
% briefly explain how VirtTAS works
% TODO extend
Our work is built upon the two other project Florian thesis. Netkernel is another project 
which shows the feasibility but does not mention the virtualization features.


%---------------------------------------------------------------------------------------------%
% what are the main contributions of this thesis
% TODO extend
We have the following contribution.
\begin{itemize}
    \item Extend TAS to support groups
    \item Use OVS to steer packets from TAS
    \item provide implementation which integrates TAS to OVS.
    \item evaluate.
\end{itemize}

%---------------------------------------------------------------------------------------------%
% structure of next sections
% TODO extend
The rest of this thesis is organized in this order.




