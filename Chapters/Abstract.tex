TCP stack suffer from poor cache allocation, 
costly context switches, and resource sharing across 
different cores, making it inefficient for modern 
data center networking. While different techniques 
have been proposed to address TCP poor efficiency issues, they suffer 
from fundamental implications regarding maintainability, and
applicability to virtualized environment, making operators 
and cloud tenants hesitant to deploy them. 

VirtTAS is an approach to address this problem by decoupling the 
network stack from VMs and placing it on the hypervisor to enable 
streamlined TCP packet processing. By utilizing a shared TCP fast-path,
 tenants benefit from the multiplexing advantages, resulting in 
improved efficiency of CPU cycles, and decreased maintenance cost 
for tenants.

This thesis presents an enrichment to Virt-TAS with 
virtualization features by integrating Open vSwitch 
(OVS) to provide efficient packet processing while 
offering network virtualization features on top of 
decoupling the network stack. 

We present three main contributions, including 
incorporating the concept of groups to improve the flexibility 
and scalability of Virt-TAS, leveraging OVS to take advantage 
of its network programmability capabilities, 
and optimizing the solution by using fast-path processing. 

We showed that VirtTAS provides \(100 \% \) increased throughput comparing to conventional
Linux kernel packet processing for short messages, while 
consuming 40\% less CPU cores, preserving precious cores 
for the main goal, the application logic.