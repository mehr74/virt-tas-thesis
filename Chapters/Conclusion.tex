In conclusion, VirtTAS provides a solution to the problem of inefficient TCP packet 
processing by decoupling the network stack from virtual machines and placing it on the 
hypervisor. By utilizing a shared TCP fast-path, VirtTAS offers multiplexing advantages, 
improved efficiency of CPU cycles, and decreased maintenance costs for tenants.

We extended VirtTAS by integrating Open vSwitch (OVS) to provide 
efficient packet processing and network virtualization features. We 
improved the flexibility and scalability of TAS by introducing groups. Furthermore, we leveraged OVS to
improve network programmability capabilities, and we optimized the solution by using fast-path processing.

The results show that VirtTAS provides a 100\% increase in throughput compared to conventional 
Linux kernel packet processing for short messages, while consuming 40\% less CPU cores. 
This ensures that valuable CPU resources can be allocated to the main application logic. 
Overall, VirtTAS with OVS integration enables efficient and flexible TCP packet processing 
in virtualized environments.